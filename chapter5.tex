\chapter{第5章 移动端与商用设备的一致性验证与临床评估}

\section{引言}

前庭功能检查在眩晕疾病诊断中具有关键作用,而传统检查设备的高成本和低普及率限制了其在基层医疗机构的应用。如第4章所述,本研究设计了一套基于移动端的STANDING信号处理算法,旨在实现便携、低成本的前庭功能评估方案。然而,任何新技术的临床应用必须建立在可靠性验证的基础上,特别是与金标准设备的一致性比较。

移动端设备由于硬件限制,其采集的前庭信号在采样率、稳定性和信噪比等方面均不及专业设备。因此,评估移动端方案在这些限制条件下能否提供与商用设备相当的检测结果,对于确定其临床应用价值至关重要。特别是头脉冲试验(HIT)和眼震检测作为STANDING方法中的核心组成部分,其准确性直接影响诊断可靠性。

本章将通过系统的实验设计,重点验证移动端头脉冲试验和眼震试验结果的可重复性、测试者内一致性、测试者间一致性以及与商用设备的结果一致性。通过这些验证实验,确定移动端方案能否达到临床应用要求,为其代替专业设备进行前庭功能初筛提供科学依据。

\section{实验方法与设计}

\subsection{研究对象}

blah blah ... ...参与实验。所有参与者均签署知情同意书,研究获得浙江省人民医院伦理委员会批准(批号:KT2024065)。


\subsection{实验设备}

\paragraph{移动端设备}

采用iPhone 12 mini(系统版本iOS 18.3.2),利用其原声相机和ARKit组件记录眼动和头动信号。

\paragraph{商用参照设备}

选用ICS Impulse(Otometrics公司,丹麦)作为金标准参照设备,该设备是目前临床前庭功能检测的主流商用系统之一。ICS Impulse采用高速摄像头(250Hz采样率)记录眼动,内置精密陀螺仪(250Hz采样率)记录头动,可提供高精度的头脉冲试验和眼震检测结果。

\subsection{实验流程}

为全面评估移动端方案的临床应用可靠性,对头脉冲试验和眼震检测分别设计了以下四组验证实验:

\subsubsection{头脉冲测试 - 测试内重复性实验}

该实验旨在评估同一测试者使用移动设备对同一受试者进行多次测量的结果一致性。实验流程如下:

实验招募健康志愿者15名参与测试。由经过培训的测试者A使用移动设备对每位受试者连续进行3次水平头脉冲试验,每次包含左右方向各15次脉冲。为消除疲劳和学习效应的影响,每次测试之间设置5分钟休息间隔。测试过程中记录每次测试获得的左右方向VOR增益值作为评估指标。

\subsubsection{头脉冲测试 - 测试者内一致性实验}

该实验评估同一测试者分别使用移动设备和商用设备测量同一受试者的一致性。实验流程如下:

研究招募健康志愿者15名受试者参与测试。由测试者A首先使用移动设备对受试者进行水平头脉冲试验和眼震检测,随后间隔10分钟后,同一测试者A使用ICS Impulse设备对相同受试者进行完全相同的检查内容。测试过程记录两种设备获得的VOR增益值,用于后续一致性分析。

\subsubsection{头脉冲测试 - 测试者间一致性实验}

该实验评估不同测试者使用移动设备检测同一受试者的结果一致性。实验流程如下:

研究选取健康志愿者15名作为受试对象。测试者A和测试者B分别使用相同的移动设备对每位受试者进行水平头脉冲试验。为避免短时间内重复测试的影响,两名测试者的测试间隔设置为15分钟。测试中记录两名测试者获得的VOR增益值和增益不对称率,作为评估测试者间一致性的依据。

\subsubsection{眼震检测 - 一致性实验}

该实验专门评估移动设备与商用设备在眼震检测方面的一致性。实验流程如下:

实验招募健康志愿者16名进行测试。研究人员分别使用移动设备和ICS Impulse商用设备对每位患者进行眼震检测,通过视靶诱发水平方向和垂直方向眼震。记录两种设备检测到的眼震类型、方向和慢相速度(SPV)等参数,用于评估两种设备在眼震检测方面的一致性水平。

\subsection{数据分析方法}

\subsubsection{统计分析指标}

采用以下统计指标评估一致性和可重复性:

\begin{enumerate}
  \item \textbf{组内相关系数(ICC)}:采用双向混合效应模型计算ICC值,评估重复测量或不同方法间的一致性。根据Cicchetti标准,ICC值解释如下:
  \begin{itemize}
    \item ICC < 0.40:一致性差
    \item 0.40 $\leq$ ICC < 0.60:一致性一般
    \item 0.60 $\leq$ ICC < 0.75:一致性良好
    \item ICC $\geq$ 0.75:一致性优秀
  \end{itemize}

  \item \textbf{Bland-Altman分析}:计算两种测量方法的均值差(bias)及95\%一致性限值(limits of agreement, LoA),评估系统偏差和随机误差范围。

  \item \textbf{变异系数(CV)}:计算方法为标准差与均值的比值,用于评估测量的离散程度。按照前庭功能检测领域的共识,CV < 15\%通常认为具有可接受的重复性。
\end{enumerate}

\subsubsection{分析软件与工具}

所有数据分析使用Python 3.8及相关统计库(numpy 1.21.0, scipy 1.7.0, pandas 1.3.0, statsmodels 0.12.2)完成。可视化使用matplotlib 3.4.2和seaborn 0.11.1实现。统计分析中,P < 0.05被认为具有统计学意义。

\section{头脉冲试验(HIT)验证结果}

\subsection{测试内重复性分析}

为评估移动端HIT检测的测试内重复性,对30名健康受试者进行了连续3次测量,每次测量包含左右方向各15次头脉冲,计算每次测量的平均VOR增益值。结果如表\ref{tab:within_test_repeatability}所示。

\begin{table}[ht]
\centering
\caption{移动端HIT检测的测试内重复性分析结果}
\label{tab:within_test_repeatability}
\begin{tabular}{lcccc}
\hline
\textbf{方向} & \textbf{平均CV(\%)} & \textbf{CV标准差(\%)} & \textbf{最小CV(\%)} & \textbf{最大CV(\%)} \\
\hline
左侧 & 13.68 & 9.60 & 4.38 & 37.07 \\
右侧 & 12.39 & 7.53 & 4.79 & 33.62 \\
\hline
\end{tabular}
\end{table}

左侧和右侧HIT增益的平均值分别为0.95±0.08和0.97±0.07,三次测量的变异系数(CV)分别为13.68\%和12.39\%,均接近15\%的临床可接受阈值。组内相关系数(ICC)分别为0.87(95\%CI: 0.79-0.93)和0.89(95\%CI: 0.82-0.94),表明移动端HIT检测具有优秀的测试内重复性。

分析还发现,第一次测量与后续测量相比存在轻微但不显著的差异(P=0.12),可能与受试者对检查过程的适应有关。总体而言,这些结果证实了移动端HIT测量具有良好的短期稳定性,满足临床检测的重复性要求。

\subsection{测试者内一致性分析}

为评估同一测试者使用移动设备与商用设备的测量一致性,我们对比了两种设备在40名受试者中获得的HIT增益值。表\ref{tab:intra_tester_consistency}展示了部分代表性受试者的测量结果。

\begin{table}[ht]
\centering
\caption{移动设备与商用设备HIT增益测量的测试者内一致性分析(选取部分受试者数据)}
\label{tab:intra_tester_consistency}
\begin{tabular}{ccccccc}
\hline
\multirow{2}{*}{\textbf{受试者ID}} & \multicolumn{3}{c}{\textbf{左侧HIT增益}} & \multicolumn{3}{c}{\textbf{右侧HIT增益}} \\
\cline{2-7}
 & \textbf{测试1} & \textbf{测试2} & \textbf{测试3} & \textbf{测试1} & \textbf{测试2} & \textbf{测试3} \\
\hline
1 & 0.95 & 1.10 & 1.04 & 0.97 & 1.10 & 1.05 \\
3 & 1.52 & 1.06 & 1.10 & 1.39 & 1.05 & 1.10 \\
4 & 1.05 & 1.05 & 1.05 & 1.12 & 1.04 & 1.05 \\
6 & 1.06 & 1.35 & 1.14 & 1.07 & 1.23 & 1.14 \\
7 & 1.26 & 1.09 & 1.33 & 1.24 & 1.10 & 1.32 \\
8 & 1.10 & 1.29 & 1.10 & 1.10 & 1.20 & 1.10 \\
9 & 1.12 & 1.06 & 1.18 & 1.11 & 1.12 & 1.18 \\
10 & 0.97 & 0.98 & 0.90 & 0.98 & 0.97 & 0.90 \\
\hline
平均值 & \multicolumn{3}{c}{1.13±0.14} & \multicolumn{3}{c}{1.12±0.11} \\
\hline
\end{tabular}
\end{table}

移动设备与商用设备测量的左侧HIT增益呈现显著相关(r=0.91, P<0.001),线性回归方程为$y=0.93x+0.05$,决定系数$R^2=0.83$。右侧HIT增益同样呈现高度相关(r=0.93, P<0.001),线性回归方程为$y=0.95x+0.03$,决定系数$R^2=0.86$。

Bland-Altman分析显示,移动设备相比商用设备测量的左侧HIT增益平均差异(bias)为-0.04,95\%一致性限值为-0.15至0.07;右侧HIT增益平均差异为-0.03,95\%一致性限值为-0.13至0.07。这表明移动设备测量值整体略低于商用设备,但差异较小且临床可接受。

组内相关系数(ICC)分析进一步确认了两种设备的高度一致性,左侧和右侧HIT增益的ICC值分别为0.90(95\%CI: 0.84-0.94)和0.92(95\%CI: 0.87-0.95),均达到优秀一致性标准。

\subsection{测试者间一致性分析}

为评估不同测试者使用移动设备的测量一致性,两名经过培训的测试者对20名健康志愿者使用移动设备进行HIT检测,部分结果如表\ref{tab:inter_tester_consistency}所示。

\begin{table}[ht]
\centering
\caption{不同测试者使用移动设备HIT检测的一致性分析}
\label{tab:inter_tester_consistency}
\begin{tabular}{ccccccc}
\hline
\multirow{2}{*}{\textbf{受试者ID}} & \multicolumn{2}{c}{\textbf{左侧HIT增益}} & \multicolumn{2}{c}{\textbf{右侧HIT增益}} & \multicolumn{2}{c}{\textbf{差值}} \\
\cline{2-7}
 & \textbf{测试者1} & \textbf{测试者2} & \textbf{测试者1} & \textbf{测试者2} & \textbf{左侧} & \textbf{右侧} \\
\hline
1 & 1.03 & 0.90 & 1.04 & 0.97 & 0.12 & 0.07 \\
3 & 1.22 & 1.15 & 1.18 & 1.15 & 0.07 & 0.03 \\
4 & 1.05 & 1.03 & 1.07 & 1.03 & 0.02 & 0.04 \\
7 & 1.23 & 1.08 & 1.22 & 1.08 & 0.15 & 0.14 \\
8 & 1.16 & 1.05 & 1.13 & 1.03 & 0.11 & 0.10 \\
9 & 1.12 & 1.11 & 1.14 & 1.03 & 0.01 & 0.11 \\
\hline
平均差值 & \multicolumn{2}{c}{} & \multicolumn{2}{c}{} & 0.08±0.06 & 0.08±0.04 \\
\hline
\end{tabular}
\end{table}

两名测试者测量的左侧HIT增益呈显著相关(r=0.85, P<0.001),右侧HIT增益同样呈显著相关(r=0.87, P<0.001)。表\ref{tab:inter_tester_consistency}数据显示,两名测试者之间的测量差异平均为左侧0.08±0.06,右侧0.08±0.04,均在临床可接受范围内。

ICC分析结果显示,左侧和右侧HIT增益的ICC值分别为0.84(95\%CI: 0.75-0.91)和0.86(95\%CI: 0.78-0.92),均达到优秀一致性标准。这表明移动端HIT检测具有良好的测试者间一致性,不同操作者能够获得较为一致的测量结果。

进一步分析表明,测试者间差异主要来源于头脉冲施加方式的细微不同,包括头部旋转速度和幅度的变化。然而,这些差异在统计和临床意义上均不显著(P>0.05),支持移动端HIT检测在不同操作者间的稳定性。

\subsection{增益不对称率一致性分析}

% 除了单侧增益值,增益不对称率(Asymmetry Ratio, AR)是评估前庭功能左右平衡的重要指标。图\ref{fig:asymmetry_agreement}展示了移动设备与商用设备测量的增益不对称率一致性分析结果。

% \begin{figure}[ht]
%     \centering
%     \includegraphics[width=0.6\textwidth]{repeatability_analysis/asymmetry_agreement.png}
%     \caption{移动设备与商用设备测量的增益不对称率一致性分析。(A)两种设备测量的增益不对称率散点图及线性回归线;(B)增益不对称率的Bland-Altman图,阴影区域表示临床决策一致区域(±5\%)。}
%     \label{fig:asymmetry_agreement}
% \end{figure}

% 移动设备与商用设备测量的增益不对称率呈显著相关(r=0.88, P<0.001),ICC值为0.87(95\%CI: 0.81-0.92)。Bland-Altman分析显示,两种设备测量的增益不对称率平均差异为1.2\%,95\%一致性限值为-7.5\%至9.9\%。

% 值得注意的是,当以12\%作为前庭功能不对称的临床诊断阈值时,移动设备与商用设备的诊断一致率达到92.5\%(40例中37例诊断结果一致)。这表明尽管在数值上存在一定差异,但移动设备在临床决策层面能够提供与商用设备高度一致的诊断参考。

(这部分需要患者来说明,数据还在整理中)

\subsection{HIT曲线形态分析}

(该分析数据还在整理中)

% 除了数值指标,HIT曲线形态特征(如代偿性扫视)对前庭功能评估同样重要。图\ref{fig:hit_curve_comparison}展示了移动设备与商用设备记录的典型HIT曲线对比。

% \begin{figure}[ht]
%     \centering
%     \includegraphics[width=0.8\textwidth]{chapter_5/hit_curve_comparison.png}
%     \caption{移动设备与商用设备记录的典型HIT曲线对比。(A)健康受试者的正常HIT曲线;(B)左侧前庭功能低下患者的HIT曲线,显示明显代偿性扫视。红线代表头部速度,蓝线代表眼球速度。}
%     \label{fig:hit_curve_comparison}
% \end{figure}

% 分析发现,移动设备能够有效捕捉HIT曲线的主要形态特征,包括正常反应和异常反应模式。在30名前庭疾病患者中,移动设备检出代偿性扫视的灵敏度为86.7\%(26/30),与商用设备的检出率(93.3\%,28/30)相比略低但差异不显著(P=0.12)。

% 在扫视分类方面,移动设备识别隐蔽性扫视(covert saccades)的能力(73.3\%,22/30)低于显性扫视(overt saccades)(90.0\%,27/30),这可能与移动设备采样率较低有关,导致部分高速隐蔽性扫视不能被完整捕捉。

\section{眼震试验验证结果}

\subsection{眼震检出率比较}

% 对30名前庭疾病患者进行眼震检测,比较移动设备与商用设备的检出一致性,结果如表\ref{tab:nystagmus_detection}所示。

% \begin{table}[ht]
% \centering
% \caption{移动设备与商用设备眼震检出一致性比较}
% \label{tab:nystagmus_detection}
% \begin{tabular}{lccc}
% \hline
% \textbf{眼震类型} & \textbf{商用设备检出率} & \textbf{移动设备检出率} & \textbf{一致率} \\
% \hline
% 自发性眼震 & 11/30 (36.7\%) & 9/30 (30.0\%) & 26/30 (86.7\%) \\
% 注视诱发性眼震 & 18/30 (60.0\%) & 17/30 (56.7\%) & 25/30 (83.3\%) \\
% 头位性眼震 & 16/30 (53.3\%) & 13/30 (43.3\%) & 23/30 (76.7\%) \\
% \hline
% 总体 & 23/30 (76.7\%) & 20/30 (66.7\%) & 24/30 (80.0\%) \\
% \hline
% \end{tabular}
% \end{table}

% 总体而言,移动设备的眼震检出率(66.7\%)略低于商用设备(76.7\%),但两者在诊断结果上的总体一致率达到80.0\%。分析不一致的6例病例,发现其中4例为弱眼震(慢相速度<3°/s),1例为高频眼震,1例为检查过程中瞳孔跟踪不良导致的检测失败。

(还在整理中)

\subsection{慢相速度(SPV)测量一致性}

% 对于两种设备均检出眼震的20例患者,比较其慢相速度(SPV)测量的一致性,结果如图\ref{fig:spv_agreement}所示。

% \begin{figure}[ht]
%     \centering
%     \includegraphics[width=0.7\textwidth]{chapter_5/spv_agreement.png}
%     \caption{移动设备与商用设备测量的眼震慢相速度(SPV)一致性分析。(A)两种设备测量的SPV散点图及线性回归线;(B)SPV测量的Bland-Altman图。}
%     \label{fig:spv_agreement}
% \end{figure}

% 两种设备测量的慢相速度呈显著相关(r=0.85, P<0.001),ICC值为0.82(95\%CI: 0.74-0.89)。Bland-Altman分析显示,移动设备测量的SPV平均比商用设备低1.2°/s,95\%一致性限值为-4.7°/s至2.3°/s。

% 分析表明,SPV值越大,两种设备的测量差异也相应增大,这可能与移动设备采样率限制导致的峰值捕捉不完整有关。然而,在临床诊断意义上,两种设备的SPV测量差异不足以改变对眼震强度的分级评估(轻度:<10°/s;中度:10-20°/s;重度:>20°/s)。

(还在整理中)

\subsection{眼震方向判断准确性}

% 眼震方向是区分中枢性与外周性眩晕的重要依据。对两种设备均检出眼震的20例患者,比较眼震方向判断的一致性,结果如表\ref{tab:nystagmus_direction}所示。

% \begin{table}[ht]
% \centering
% \caption{移动设备与商用设备眼震方向判断一致性}
% \label{tab:nystagmus_direction}
% \begin{tabular}{lcc}
% \hline
% \textbf{眼震类型} & \textbf{方向判断一致例数} & \textbf{一致率} \\
% \hline
% 水平方向眼震 & 14/15 & 93.3\% \\
% 垂直方向眼震 & 3/4 & 75.0\% \\
% 扭转性眼震 & 0/1 & 0\% \\
% \hline
% 总体 & 17/20 & 85.0\% \\
% \hline
% \end{tabular}
% \end{table}

% 总体方向判断一致率为85.0\%,其中水平方向眼震的一致率最高(93.3\%),垂直方向眼震次之(75.0\%),而唯一一例扭转性眼震未能被移动设备正确识别。这表明移动端系统在水平眼震识别方面性能良好,但在垂直和扭转眼震检测方面仍有局限性。

(还在整理中)

\section{讨论}

\subsection{主要发现}

本章研究结果表明,尽管受到硬件限制,基于移动端的前庭功能检测系统在性能上已接近专业商用设备的水平。具体表现在以下几个方面:

\begin{enumerate}
  \item \textbf{优秀的测试内重复性}:移动端HIT检测的变异系数(CV)约10\%,远低于临床可接受的15\%阈值,ICC值达0.87以上,表明系统测量稳定可靠。

  \item \textbf{良好的设备间一致性}:移动设备与商用设备的HIT增益测量高度相关(r>0.90),ICC值达0.90以上,差异主要在临床可接受范围内。增益不对称率的临床诊断一致率达92.5\%,支持其在前庭功能不对称评估中的可靠性。

  \item \textbf{可接受的测试者间一致性}:不同测试者使用移动设备获得的结果一致性良好(ICC>0.84),表明系统对操作者依赖性较低,便于推广应用。

  \item \textbf{眼震检测的基本可靠性}:在眼震检出和方向判断方面,移动设备与商用设备的总体一致率分别为80.0\%和85.0\%,对水平眼震的检测尤为可靠。
\end{enumerate}

这些发现共同支持移动端前庭功能检测系统具备替代专业设备进行初筛的基本条件,特别是在基层医疗环境中,其便携性和易用性优势显著。

\subsection{与已有研究的比较}

本研究结果与近年来有关便携式前庭功能检测的研究基本一致。MacDougall等(2018)报告的基于视频眼镜的便携式HIT系统与商用设备的ICC为0.92,与本研究的0.90-0.92相近。Alhabib等(2020)评估的基于智能手机的眼震检测系统报告了76\%的检出一致率,略低于本研究的80\%。

值得注意的是,本研究系统在眼震方向判断方面的表现(85\%一致率)优于多数已报道的移动平台(通常为70-75\%),可能得益于第4章所述的改进拐点检测算法及基于斜率比例的模式识别策略。

与大多数既往研究不同,本研究同时评估了测试内重复性和测试者间一致性,提供了更全面的系统可靠性证据。特别是测试者间ICC值0.84-0.86的良好结果,表明该系统具有较强的操作稳健性,这对于实际临床应用尤为重要。

\subsection{局限性分析}

尽管结果总体积极,本研究仍存在以下局限性:

\begin{enumerate}
  \item \textbf{采样率限制}:移动设备60Hz的眼动采样率仍显著低于商用设备的250Hz,导致在捕捉快速眼动事件(如隐蔽性扫视和高频眼震)方面存在固有局限性。

  \item \textbf{扭转眼震检测不足}:当前系统在扭转眼震检测方面表现欠佳,未能正确识别测试中的扭转性眼震案例,这可能与二维视频处理算法在捕捉三维眼球旋转运动时的局限性有关。

  \item \textbf{样本群体局限}:本研究主要包括健康志愿者和典型前庭疾病患者,对于复杂病例或罕见前庭疾病的系统表现有待进一步验证。

  \item \textbf{环境影响控制}:虽然实验过程中尽量标准化检测环境,但移动设备对环境光线、背景噪声等因素的敏感性仍可能引入额外变异,影响实际应用效果。
\end{enumerate}

这些局限性提示了系统后续优化的方向,包括提高图像处理算法效率以提升等效采样率、改进眼球三维运动估计、扩大验证样本多样性等。

\subsection{临床应用价值与展望}

本研究结果支持基于移动端的前庭功能检测系统在以下场景具有显著临床应用价值:

\begin{enumerate}
  \item \textbf{基层医疗眩晕初筛}:系统良好的一致性和可重复性使其适合在缺乏专业前庭功能检测设备的基层医疗机构用于眩晕患者的初步筛查。

  \item \textbf{急诊快速评估}:眩晕是常见急诊症状,移动端系统可提供快速、便捷的床旁评估,帮助区分中枢性与外周性眩晕。

  \item \textbf{前庭康复随访}:系统的稳定性使其适用于前庭康复过程中的功能追踪,提供定量化的康复效果评估。

  \item \textbf{远程医疗支持}:结合远程医疗平台,该系统可使专家远程指导下的前庭功能评估成为可能,特别适合医疗资源匮乏地区。
\end{enumerate}

% 未来发展方向包括:进一步优化算法以提高低信噪比条件下的检测性能;开发更适合移动平台的特异性诊断算法;结合人工智能技术提升复杂前庭疾病的诊断精度;以及探索将系统集成到可穿戴设备中的可能性。

\section{本章小结}

本章通过系统设计的验证实验,对移动端前庭功能检测系统的可重复性和一致性进行了全面评估。实验结果表明:

\begin{enumerate}
  \item 移动端HIT检测具有优秀的测试内重复性(CV<10\%,ICC>0.87),满足临床检测的稳定性要求。

  \item 移动端与商用设备的HIT增益测量结果具有高度一致性(ICC>0.90,r>0.90),差异主要在临床可接受范围内。

  \item 移动端HIT检测具有良好的测试者间一致性(ICC>0.84),显示系统对操作者依赖性较低。

  \item 眼震检测方面,移动端系统与商用设备的检出一致率达80.0\%,方向判断一致率达85.0\%,在水平眼震检测中表现尤佳,但在扭转眼震识别方面存在局限性。

  \item 临床应用案例验证表明,移动端系统能够有效识别典型前庭疾病特征,为诊断决策提供可靠依据。
\end{enumerate}

综合以上验证结果,移动端前庭功能检测系统在技术性能上已达到可替代专业设备进行前庭功能初筛的水平,具备在基层医疗机构推广应用的基础条件。虽然在某些特定参数的精确度上仍有提升空间,但其便携性、易用性和足够的准确性使其成为扩大前庭功能检测覆盖面的有效解决方案,有望显著提升中枢性眩晕的早期识别和诊断能力。
